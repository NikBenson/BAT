\lipsum[5]

\subsection{Anforderungen}\label{subsec:anforderungen2}
\lipsum[5]

\subsection{Vergleich der Transformationsschritte}\label{subsec:vergleich-der-transformationsschritte}
\begin{figure}
    \begin{framed}
        \lstinputlisting[label={lst:c-input},language=txt,nolol,numbers=none,backgroundcolor={}]{../assets/code/c/input.txt}
    \end{framed}
    \begin{tikzpicture}
        \path[draw=none] (0,0) -- (\textwidth,0);
        \draw[->] (0.5\textwidth,1) -- (0.5\textwidth,0) node[midway,above,right] {Lex};
    \end{tikzpicture}
    \begin{framed}
        \lstinputlisting[label={lst:c-tokens},language=tokens,nolol,numbers=none,backgroundcolor={}]{../assets/code/c/tokens.txt}
    \end{framed}
    \begin{tikzpicture}
        \path[draw=none] (0,0) -- (\textwidth,0);
        \draw[->] (0.5\textwidth,1) -- (0.5\textwidth,0) node[midway,above,right] {\acs{YACC}};
    \end{tikzpicture}
    \begin{framed}
        \includegraphics[width=\textwidth]{../assets/img/diagrams/ard_player_c_ast.mmd}
    \end{framed}
    \begin{tikzpicture}
        \path[draw=none] (0,0) -- (\textwidth,0);
        \draw[->] (0.5\textwidth,1) -- (0.5\textwidth,0) node[midway,above,right] {Selbstgeschriebener C Compiler};
    \end{tikzpicture}
    \begin{framed}
        \lstinputlisting[label={lst:c-output},language=JavaScript,nolol,numbers=none,backgroundcolor={}]{../assets/code/c/output.txt}
    \end{framed}
    \caption{Schritte des Compilierens der PlayerConfig Sprache mit Lex \& \acs{YACC}}
    \label{fig:ard-player-c-stages}
\end{figure}
\lipsum[5]

\subsection{Vergleich der Code--Generatoren}\label{subsec:vergleich-der-code--generatoren}
\lipsum[5]

\subsection{Vor-- und Nachteile der klassischen Tools}\label{subsec:vor---und-nachteile-der-klassischen-tools}
\lipsum[5]

\subsection{Vor-- und Nachteile von \acs{MPS}}\label{subsec:vor---und-nachteile-von-mps}
\lipsum[5]
