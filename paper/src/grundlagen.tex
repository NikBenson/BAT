In diesem Kapitel werden die grundsätzlichen Definitionen und eine historische einordnung sowie ein Überblich über die aktuellen Möglichkeiten von \ac{DSL} geschaffen.

\subsection{\aclp{DSL}}\label{subsec:domain-specific-languages}
Die erste Frage, die es zu klären gibt, ist: Was sind eigentlich \ac{DSL}?
\begin{displayquote}[\cite{jetbrains-sro-no-dateC}]
    A \ac{DSL} is a programming language with a higher level of abstraction optimized for a specific class of problems.
    A \ac{DSL} uses the concepts and rules from the field or domain.
\end{displayquote}
Laut JetBrains definieren sich \acp{DSL} also über ihr Sprachdesign, welches Konzepte einer spezifischen, fachlichen Domäne beinhaltet.
Wikipedia definiert es etwas anders:
\begin{displayquote}[\cite{wikipedia-contributors-2024C}]
    A \ac{DSL} is a computer language specialized to a particular application domain.
    This is in contrast to a \ac{GPL}, which is broadly applicable across domains.
\end{displayquote}
Hier werden \acp{DSL} als gegenteil von \acp{GPL} genannt, welche zusammen die Programmiersprachen ausmachen.
Diese sind im Wörterbuch definiert als:
\begin{displayquote}[\cite{unknown-author-no-date}]
    Code of reserved words and symbols used in computer programs, which give instructions to the computer on how to accomplish certain computing tasks.
\end{displayquote}
Also eine Kodierung, welche Anweisungen an einen Computer gibt.

Zusammenfassend lässt sich sagen, dass \acp{DSL} Kodierungen sind, die fachspezifisch von einer Menge von Schlüsselwörtern und strukturierten, dynamischen Inhalt auf eine von (mindestens) einem Computer--(Programm) verstandene Eingabe--(Datei) abbildet.
Dabei stehen fachspezifische Paradigmen vor den klassischen Programmierparadigmen~\autocite{wikipedia-contributors-2024D} im Zentrum des Sprachdesigns.

Das wohl bekannteste Beispiel ist \ac{SQL}.~\autocite{unknown-author-2023}
\ac{SQL} ist eine durch die \ac{ISO} genormte \ac{DSL}; spezifischer eine sogenannte \textit{query language}.
Auf Deutsch Abfragesprachen genannt, dienen diese \acp{DSL}, der interaktionen mit \acp{DBMS}.
Die Domäne ist in diesem Fall also die Datenhaltung \textit{strukturierter} Daten.

Da es etliche \acp{DBMS} gibt, welche dem \ac{SQL}--Standard folgen, gibt es für \ac{SQL} auch etliche Compiler beziehungsweise Interpreter.
Diese Compiler sind dabei stark mit dem jeweiligen \ac{DBMS} verwoben.
Es gibt also keine einheitliche Codebasis.
Dies wäre mit bekannten Tools des Compilerbaus, namentlich \textbf{Lex} \& \textbf{YACC}.
Diese werden im folgenden Kapitel behandelt.

\subsection{Compilerbau}\label{subsec:compilerbau}

\begin{wrapfigure}{rH}{0.5\textwidth}
    \begin{center}
        \includegraphics[width=0.48\textwidth]{../assets/img/diagrams/compiler_phases.mmd}
    \end{center}
    \caption{Phasen der Compilerpipeline~\autocite{aho-2006}}
    \label{fig:cpmpiler-phases}
\end{wrapfigure}
\lipsum[5]

\subsubsection{Tokenizer --- Lex}
\lipsum[5]

\subsubsection{Parser --- \acs{YACC}}
\lipsum[5]

\subsubsection{Compiler --- C}
\lipsum[5]

\subsection{\acl{MPS}}\label{subsec:meta-programming-system}
\lipsum[5]
