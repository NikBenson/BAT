\addcontentsline{toc}{section}{Abkürzungsverzeichnis}
\section*{Abkürzungsverzeichnis}
\begin{acronym}[POSIX]
    \acro{ARD}{Arbeitsgemeinschaft --- der öffentlich--rechtlichen Rundfunkanstalten --- der Bundesrepublik Deutschland}
    \acro{AST}{Abstract Syntax Tree}

    \acro{CD}{Continuous Deployment}

    \acro{CFG}{context--free grammar}

    \acro{CI}{Continuous Integration}

    \acro{CLI}{Comand Line Interface}
    \acroplural{CLI}[CLI]{Comand Line Interfaces}

    \acro{DBMS}{Database Management System}
    \acroplural{DBMS}[DBMS]{Database Management Systems}

    \acro{DSL}{Domain--Specific Language}
    \acroplural{DSL}[DSL]{Domain--Specific Languages}

    \acro{GCC}{GNU Compiler Collection}

    \acro{GPL}{General--Purpose Language}
    \acroplural{GPL}[GPL]{General--Purpose Languages}

    \acro{ID}{Identifikation}

    \acroplural{ID}[ID]{Identifikationen}

    \acro{IDE}{Integrated Development Environment}

    \acro{ISO}{International Organization for Standardization}

    \acro{IT}{Information Technology}

    \acro{JSON}{JavaScript Object Notation}

    \acro{LALR}{lookahead, left--to--right, rightmost deriviation}

    \acro{LL}{left--to--left}

    \acro{LLM}{Large Language Model}
    \acroplural{LLM}[LLM]{Large Language Models}

    \acro{LOP}{Language Oriented Programming}

    \acro{LR}{left-to-right, rightmost derivation in reverse}

    \acro{LLVM}{Low Level Virtual Machine}

    \acro{MPS}{Meta Programming System}

    \acro{MVP}{Minimal Viable Product}

    \acro{OOP}{Objektorientierte Programmierung}

    \acro{POSIX}{Portable Operating System Interface}

    \acro{RegEx}{Regulärer Ausdruck}
    \acroplural{RegEx}[RegEx]{Reguläre Ausdrücke}

    \acro{SSOT}{single source of truth}

    \acro{SoC}{Seperation of Concerns}

    \acro{SQL}{Structured Query Language}

    \acro{SWR}{Südwestrundfunk}

    \acro{UI}{User Interface}
    \acroplural{UI}[UI]{User Interfaces}

    \acro{VCS}{Versionsverwaltung}

    \acro{XML}{Extensible Markup Language}

    \acro{YACC}{Yet Another Compiler Compiler}
\end{acronym}