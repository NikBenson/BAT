Es ist nicht zu bestreiten, dass der Prototyp in seinem momentanen zustand noch nicht bereit zur nutzung durch Domainexperten ist.
Als Entwickler, der in seiner \ac{IDE} zuhause ist, spezifisch IntelliJ, lässt sich die Sprache allerdings bereits gut navigieren.

Um zu schauen, ob der Prototyp zukünftig theoretisch und praktisch anwendbar ist, wurden zwei Designer befragt.
Hierzu wurden einige, wenige Transformatoren exemplarisch umgesetzt, um den Designern eine Hands--On--Erfahrung bieten zu können.

\subsection{Befragungen}\label{subsec:befragungen}
Befragt wurden ein Designer bei \ac{netTrek} und ein Designer des \ac{SWR}.
Diese wurden ausgewählt, da sie sich in der direkten Zielgruppe beziehungsweise in den beiden Teams mit dem größten potenzial einer ersten nutzung befinden.

Bei \ac{netTrek} hat dies der Julian Hültenschmidt übernommen.
Er ist der Lead--Designer und hat damit die nötige expertise.
\begin{displayquote}[\autoref{subsec:julian-hultenschmidt}]
    In der aktuellen Entwicklungsphase kann die Bearbeitung von Tokens noch etwas komplex sein.
    Die Vorteile sind jedoch bereits jetzt offensichtlich: Designer können dynamische Elemente mit benutzerdefinierten Variablen und Werten erstellen, was ihre kreative Freiheit und Kontrolle über das Endprodukt erhöht.
\end{displayquote}
Er sieht klare Vorteile im Kommunikationsfluss, auch wenn er die Software, so wie sie ist noch nicht benutzen kann.
Des Weiteren sieht er sich in der Lage, sollte die Software in zukunft der vollen Spezifikation entsprechen, diese mit seinem Domänenwissen nutzen zu können.
Sein größter Kritikpunkt ist, dass es aktuell eine starke Moderation durch den Entwickler bei der Benutzung bedarf und sieht dies auch nicht vollständig schwinden, wenn die Software vollständig der Spezifikation entspricht.

Beim \ac{SWR} hat Pascal Börger den Prototypen evaluiert.
Er ist \acs{UI}-- und \acs{UX}--Designer beim \ac{SWR}.
Hier agiert we vor allem im Team App zusammen mit \ac{netTrek}.
\begin{displayquote}
    Für mich als Designer bietet MPS viele Vorteile bei der Erstellung und Wartung einer App.
    Für mich eines der Knackpunkte bei der Zusammenarbeit mit Entwicklern ist eine gemeinsame Sprache und eine Single Source of truth zu finden.
    Beides wird hier gefördert.
    Ich kann selbst den Code schreiben, um meine Tokens zu erstellen und kann somit nicht nur in Figma, sondern auch auf Code-Basis das Designsystem warten.
    So fällt sehr viel Mehraufwand im Team weg und die Fehleranfälligkeit sinkt rapide!
    Auch trägt dies dem Verständnis in der Zusammenarbeit bei.
    Ein Win-Win für das ganze Team!
\end{displayquote}
Für ihn ist die \ac{SSOT} ein besonders wichtiger Aspekt.
Auch er ist vom Konzept des Prototypen überzeugt, kann ihn im aktuellen Zustand allerdings nicht sinnvoll nutzen.

Beide \ldots

\subsection{Zukunftsperspektiven}\label{subsec:zukunftsperspektive}
\lipsum[5]
