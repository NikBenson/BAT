Nach der Betrachtung aus Entwickler-- sowie Designer--sicht soll nun die Sinnhaftigkeit aus finanzieller Sicht betrachtet werden.
Es soll die Frage beantwortet werden, inwiefern \acp{DSL} mithilfe von \ac{MPS} wirtschaftlich sein können.

Dazu wird das komplexe Beispiel des Design--System--Creators genutzt, um eine begründete Vermutung anzustellen.
Die möglichen Ergebnisse sind, dass es sinnvoll sein kann, oder falls es in diesem Beispiel nicht sinnvoll ist, dass diese Arbeit keine Aussage dazu treffen kann.

\subsection{Zeitaufwand}\label{subsec:zeitaufwand}
\begin{table}[ht]
    \centering
    \begin{tabular}{|l|c|}
        \hline
        Aufgabe                                & Zeit [Stunden] \\
        \hline
        \hline
        Lernen \& \ac{JSON}--Beispiel \ac{MPS} & 40             \\
        \hline
        Arbeit am komplexen Beispiel           & 112            \\
        \hline
        \ac{JSON}--Beispiel \ac{POSIX}         & 16             \\
        \hline
        \hline
        Gesamt                                 & 168            \\
        \hline
    \end{tabular}
    \caption{Zeitbuchungen: Zusammenfassung (Siehe \autoref{appendix:zeitbuchungen})}
    \label{tab:zeitbuchungen-short}
\end{table}
Zunächst ist der Kosten--Nutzen--Faktor zu betrachten.
Dazu sind in \autoref{tab:zeitbuchungen-short} die Zeiten aufgeführt, welche ich mit den verschiedenen Projekten verbracht habe.

Eine minimale Version, welche die gröbsten Mängel der aktuellen Version glattbügelt, sollte in weiteren zwei Wochen umgesetzt werden können.
Entsprechend hat das Projekt Design--System--Creator einen ungefähren Aufwand von 24 Tagen\footnote{Es wird von einem Arbeitstag mit 8 Stunden ausgegangen.}.

Zusätzlich dazu haben die anderen Projekte 7 Tage in Anspruch genommen.
Da lediglich das Projekt Design--System--Creator betrachtet wird, werden diese zunächst nicht mitberechnet.
Sie werden allerdings später wieder aufgegriffen, da sie eine Aussage über die Lernkurve treffen können.

Bei einem angenommenen Tagessatz von 1200~\euro, basierend auf~\cite{metrics-germany-gmbh-2021}, entspricht dies einer Summe von rund 28.800~\euro.
Dies ist eine Summe, die in den meisten Unternehmen gut gerechtfertigt werden muss.

\subsection{Komplexität}\label{subsec:komplexitat}
Da \ac{MPS} noch nicht weit verbreitet ist und eine sehr steile Lernkurve hat, sind auch potenzielle Wartungsarbeiten zu beachten.
Die Komplexität war auch bei dem momentanen Entwicklungsstand des Prototyps immer wieder ein wichtiger Punkt.

Unter dieser Betrachtung alleine sollte immer mit Vorsicht darauf geachtet werden, ob \ac{MPS} die richtige Lösung für ein Problem ist.
Die komplexität macht allerdings \ac{POSIX} nicht zur besseren Lösung.
Aus \autoref{subsec:befragungen} lässt sich entnehmen, dass sich die Einschränkungen der bearbeitungsmöglichkeiten, die \ac{MPS} mehr bieten kann, als klassischer Text, durchaus auszahlen.
Außerdem ist die Komplexität ähnlich aufgebaut wie klassische Herausforderungen aus der Softwareentwicklung.
Da es keine Alternativen mit diesen Möglichkeiten gibt, ist diese Komplexität lediglich wichtig für die Lernkurve und anfallende Wartungsarbeiten.

Dennoch ist diese Komplexität sehr wichtig zu betrachten und ein Risiko, das eingegangen wird.

\subsection{Wirtschaftlichkeit}\label{subsec:wirtschaftlichkeit}
In \autoref{subsec:befragungen} konnte festgestellt werden, dass die Domainexperten, also die Designer, das Projekt gut finden.
Um dies in Zeit beziehungsweise Geld aufrechnen zu können sind die Einsparungen zu betrachten.

Zunächst wäre dabei eine Steigerung der Präzision, wie es immer der Fall ist, wenn Kommunikationswege gekürzt werden.
Der Wert davon hängt allerdings stark am Unternehmen.
Aus persönlicher Erfahrung aus dem Team--App kann bericht werden, dass Designfeinheiten manchmal auch rechtliche Konsequenzen haben können. % TODO: Citation needed
Hier wäre die NEWSZONE--App vom Format DASDING des \ac{SWR} zu nennen.
Entsprechend ist diese Präzision im Zielumfeld manchmal sehr wichtig.
Allerdings lässt sich dies schwer in Geld umrechnen.

Konkreter wird es bei der eingesparten Arbeitszeit.
Hier wären zum einen die Meetings zwischen Designern und Entwicklern, aber auch, dass einige Aufgaben von Designern übernommen werden können, die zuvor Entwickler machen mussten.
Hier könnten auch gewisse Unterschiede im Gehalt vorhanden sein.

Selbst bei einer Einsparung von einer halben Stunde pro Woche und Person (Designer und Entwickler) im Team kommt so schnell eine große Summe zustande.
Bei einem Team von 10 Leuten, die durchschnittlich einen Tagessatz von 800~\euro kosten, ist es in einem Jahr eine Ersparnis von 26.000~\euro.

Da das Produkt allerdings auch ohne Mehrkosten in weiteren Teams verwendet werden kann oder weiterverkauft werden kann, kann es sich potenziell auch schon im ersten Jahr rechnen.

Dies bestätigt auch Bayram Ünlü, \ac{CEO} bei \ac{netTrek}, in seinem Statement in \autoref{appendix:bayram-unlu}.

Alles in allem lässt sich also vermuten, dass die fertige Software in diesem Fall wirtschaftlich sein wird.
