In dieser Arbeit soll evaluiert werden, ob das Entwickeln von \acp{DSL} mit modernen Tools wirtschaftlich ist.

\subsection{Motivation}\label{subsec:motivation}
In den vergangenen Jahren ist das, was mittels Software möglich ist stark angestiegen.
Damit wachsen auch immer mehr die Anforderungen an Entwickler.
Einen Teil der Last wird dabei schon seit einiger Zeit versucht durch Low-- bzw.\ No--Code--Solutions zu umgehen.
Ein gutes Beispiel sind dabei die Menge an neuen Website--Baukästen, die zuletzt beworben wurden. \autocite{p-2024}

Mit besonders großem Erfolg werden dabei Systeme beworben, die auf spezifische Bereiche spezialisiert sind.
Beispiele wären zum Beispiel die aktuell vom DSB geförderte Vereinswebsite~\autocite{deutscher-olympischer-sportbund-ev-no-date} sowie die beiden E-Commerce Plattformen Shopware~\autocite{ag-no-date} und Shopify~\autocite{shopify-international-ltd-no-date}.

Daraus folgt die Annahme, dass das notwendige Expertenwissen zur inbetriebnahme von Systemen weiterhin gewünscht ist.
Lediglich das Expertenwissen zur Softwareentwicklung sollte reduziert werden.

Eine allgemeingültige lösung wollen aktuell \acp{LLM} schaffen.
Diesen mangelt es allerdings für viele Anwendungsfälle an Präzision.

\acp{DSL} bieten hier eine potenzielle Lösung.

\subsubsection{Aktuelle Herausforderungen}
Die Entwicklung von \acp{DSL} ist jedoch nach wie vor mit einigen Herausforderungen verbunden.
Dazu gehört die hohe Komplexität der \ac{DSL}--Entwicklung, die tiefgreifendes Wissen in Sprachtheorie, Compilerbau und Softwareentwicklung erfordert.
Es fehlt zudem an einem einheitlichen Ansatz zur Entwicklung von \acp{DSL}, der für verschiedene Anwendungsbereiche geeignet ist.
Des Weiteren sind \acp{DSL} oft nicht in bestehende Entwicklungsumgebungen integriert, was zu Inkompatibilitäten und Fragmentierung führt.

\subsubsection{Bedarf an intuitiven und effizienten \acp{DSL}}
Eine Bedarfsanalyse kann diese Arbeit leider nicht bieten, allerdings soll sie die Möglichkeiten aufzeigen.

Sprachen schnell formulieren zu können hat in jedem Fall viele Anwendungsmöglichkeiten, die aktuell weitestgehend mittels anderer technologien gelöst werden.

Häufig kommen im Enterprise Bereich hier komplizierte \acp{UI} zum Einsatz.
Ein Beispiel dafür wäre SAP\@. % TODO: Quelle hinzufügen
Hier wurde sich dafür entschieden eine parallele zwischen \acp{UI} und \acp{DSL} zu nutzen, da \ac{DSL} ab einer gewissen komplexität einfacher zu handhaben sind.

Wir nehmen an, dass der Bedarf aktuell künstlich durch weniger optimale Technologien gedeckt wird, da \acp{DSL} zu kompliziert sind.
Es ist davon auszugehen, dass \ac{DSL} wieder inder Breite eingesetzt werden, sobald die Technologie dafür hinreichend einfach zu lernen ist.

\subsubsection{Potenzial des Prototyps}
Der vorliegende Prototyp zielt darauf ab zu prüfen, ob \ac{MPS}~\autocite{jetbrains-sro-2021} die Entwicklung von \acp{DSL} bereits so weit vereinfacht hat, dass es wirtschaftlich ist, diese Technologie einzusetzen.

Des weiteren soll die Interaktion zwischen Entwicklern und Designern in dem später erläuterten konkreten Beispiel optimiert werden.
Dabei werden unnötige Copy--Paste Arbeiten unterbunden.
Der konkrete Anwendungsfall ist allerdings in dieser Arbeit zweitrangig.

\subsubsection{Persönliche Motivation}
Die Entwicklung des Prototyps ist von meiner persönlichen Motivation geleitet, die Potenziale von \acp{DSL} zu erschließen und die Entwicklung von Software zu verbessern.
Ich sehe \acp{DSL} als eine Möglichkeit, auch Laien die präzise Beschreibung von komplexen Systemen zu ermöglichen und die IT-Branche zu demokratisieren.
Durch die Förderung der \ac{DSL}--Entwicklung und die Verbreitung von Wissen über diese Technologie möchte ich zu einer Renaissance des Compilerbaus beitragen und die Wertschätzung für die Grundlagen der Softwareentwicklung stärken.

\subsection{Zielsetzung}\label{subsec:zielsetzung}
\lipsum[5]

\subsection{Aufbau der Arbeit}\label{subsec:aufbau-der-arbeit}
\lipsum[5]
