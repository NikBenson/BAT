\appendix


\section{Befragungsergebnisse}\label{sec:befragungsergebnisse}
\ldots

\subsection{Pascal Börger}\label{subsec:pascal-borger}
\ldots

\subsection{Julian Hültenschmidt}\label{subsec:julian-hultenschmidt}
\ldots

\section{Code}\label{sec:code}
Da \acs{MPS}--Code als \ac{XML} gespeichert wird, ist es nicht sinnvoll, diesen diesem Dokument beizufügen.
Deshalb wird der Code über GitHub bereitgestellt.
Das repository ist unter folgender Address abrufbar: \href{https://github.com/NikBenson/BAT}{https://github.com/NikBenson/BAT}.
Der zu betrachtende Branch ist Main und die veröffentlichte Version wird den Tag {\ttfamily v1.0.0} haben.
Der Code liegt im Ordner {\ttfamily projects/} und ist ab folgendem Hash final: \enquote{98c5cb4f9d-7673803bb8-96d8e466ec-c389dd9f1b}.


\section{Eidesstattliche Versicherung}\label{sec:eidesstattliche-versicherung}
\ldots


\section{Zeitbuchungen}\label{sec:zeitbuchungen}
\begin{table}[H]
    \centering
    \begin{tabular}{|l|c|c|}
        \hline
        Aufgabe                                & Datum      & Zeit [Stunden] \\
        \hline
        \hline
        Lernen \& \ac{JSON}--Beispiel \ac{MPS} & 19.02.2024 & 8              \\
        \hline
        Lernen \& \ac{JSON}--Beispiel \ac{MPS} & 20.02.2024 & 8              \\
        \hline
        Lernen \& \ac{JSON}--Beispiel \ac{MPS} & 21.02.2024 & 8              \\
        \hline
        Lernen \& \ac{JSON}--Beispiel \ac{MPS} & 22.02.2024 & 8              \\
        \hline
        Lernen \& \ac{JSON}--Beispiel \ac{MPS} & 23.02.2024 & 8              \\
        \hline
        Arbeit am komplexen Beispiel           & 26.02.2024 & 8              \\
        \hline
        Arbeit am komplexen Beispiel           & 27.02.2024 & 8              \\
        \hline
        Arbeit am komplexen Beispiel           & 28.02.2024 & 8              \\
        \hline
        Arbeit am komplexen Beispiel           & 29.02.2024 & 8              \\
        \hline
        Arbeit am komplexen Beispiel           & 30.02.2024 & 8              \\
        \hline
        Arbeit am komplexen Beispiel           & 29.03.2024 & 8              \\
        \hline
        Arbeit am komplexen Beispiel           & 30.03.2024 & 4              \\
        \hline
        Arbeit am komplexen Beispiel           & 01.04.2024 & 8              \\
        \hline
        Arbeit am komplexen Beispiel           & 02.04.2024 & 8              \\
        \hline
        Arbeit am komplexen Beispiel           & 03.04.2024 & 8              \\
        \hline
        Arbeit am komplexen Beispiel           & 04.04.2024 & 8              \\
        \hline
        Arbeit am komplexen Beispiel           & 05.04.2024 & 8              \\
        \hline
        Arbeit am komplexen Beispiel           & 06.04.2024 & 8              \\
        \hline
        \ac{JSON}--Beispiel \ac{POSIX}         & 11.04.2024 & 8              \\
        \hline
        \ac{JSON}--Beispiel \ac{POSIX}         & 12.04.2024 & 8              \\
        \hline
        Arbeit am komplexen Beispiel           & 15.04.2024 & 2              \\
        \hline
        Arbeit am komplexen Beispiel           & 16.04.2024 & 2              \\
        \hline
        Arbeit am komplexen Beispiel           & 15.04.2024 & 2              \\
        \hline
        Arbeit am komplexen Beispiel           & 20.04.2024 & 4              \\
        \hline
        Arbeit am komplexen Beispiel           & 22.04.2024 & 2              \\
        \hline
    \end{tabular}
    \caption{Zeitbuchungen}
    \label{tab:zeitbuchungen-long}
\end{table}

